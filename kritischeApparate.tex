% !TeX root = lfgw.tex
\chapter{Kritische Apparate setzen}
\dictum[?]{Kritik ist überall, zumal in Deutschland, nötig.}

\index{Apparat} \index{Varianten}
\label{reledmac}

\enquote{Edieren ist eine Erziehung zur Bescheidenheit [...] Es ist ferner eine Erziehung zur 
Genauigkeit, wie alle Philologie.}%
\footnote{Erich Trunz, Ein Tag aus Goethes Leben. Acht Studien zu Leben und Werk, München 1999, S.~213.}

Bezug zu TUSTEP?
Recherche zu anderen Alternativen?

\footcite[165\psqq]{rouquette:2012}

\cite{stockhausen:welten}

Mit \Package{ednotes} und \Package{reledmac} liegen zwei Pakete vor, die allen Anforderungen einer kritischen Edition gerecht werden.%
\footnote{%
	%Vergleich ednotes/ledmac: http://www.webdesign-bu.de/uwe_lueck/critedltx.html
	Auf das Paket \Package{ednotes} von Uwe Lück gehe ich hier nicht weiter ein, ein ausführlicher \TUGboat-Artikel liegt dem Paket bei. Eine Arbeit, die mit diesem Paket erstellt wurde, ist \cite{mariev:joh_ant}.}
\Package{reledmac} befindet sich seit Ende der 80er Jahren in mehr oder weniger steter Fortentwicklung, der aktuelle Maintainer Maïeul Rouquette arbeitet Fehler-Berichte oder Feature-Wünsche freundlich und effizient ab.%
\footnote{%
	Das Paket \Package{reledmac} \cite{reledmac} geht ursprünglich auf das \plainTeX{}-Paket \Package{edmac} von John Lavagnino und Dominik Wujastyk zurück. Dieses wurde ab 2003 von Peter Wilson für \LaTeX{} als \Package{ledmac} portiert und weiterentwickelt. Im Jahr 2011 hat Maïeul Rouquette das Paket übernommen und erst als \Package{eledmac}, später als \Package{reledmac} fortgeführt.}
Das Paket hat sich in mehreren publizierten Editions(groß)projekten als gleichermaßen stabil und flexibel erwiesen.%
\footnote{%
	Beispielsweise die Erlanger Athanasius-Arbeitsstelle verwendet seit zehn Jahren \Package{(re)ledmac} und hat damit bereits mehrere Bände publiziert. Dazu vgl. \cite{stockhausen:welten}. Eine (unvollständige) Liste der Editionen in den unterschiedlichsten Sprachen, die \Package{(rel)edmac} verwendet haben, findet sich unter \cite{reledmac-benutzung}.}